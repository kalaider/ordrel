% \iffalse meta-comment
%
% Copyright (C) 2020 by Alexander Vasilevsky (kalaider) <a.kalaider@yandex.ru>
% ---------------------------------------------------------------------------
% This work may be distributed and/or modified under the
% conditions of the LaTeX Project Public License, either version 1.3
% of this license or (at your option) any later version.
% The latest version of this license is in
%   http://www.latex-project.org/lppl.txt
% and version 1.3 or later is part of all distributions of LaTeX
% version 2005/12/01 or later.
%
% This work has the LPPL maintenance status `maintained'.
%
% The Current Maintainer of this work is Alexander Vasilevsky.
%
% This work consists of the files ordrel.dtx and ordrel.ins
% and the derived filebase ordrel.sty.
%
% \fi
%
% \iffalse
%<*driver>
\ProvidesFile{ordrel.dtx}

% \pkg{underscore} is loaded by l3doc unconditionally,
% we need to provide all the necessary fixes in advance
\PassOptionsToPackage{strings}{underscore}
\documentclass{l3doc}

\usepackage[T1]{fontenc}
\usepackage[utf8]{inputenc}
\usepackage[english]{babel}
\usepackage{microtype}

\usepackage{enumitem}
\usepackage{mdframed}
\usepackage{xcolor}
\usepackage{caption}
\usepackage{hyperref}
\usepackage{cleveref}

\usepackage[quiet]{ordrel}[2020/09/18]

\setlength{\parindent}{0pt}
\addtolength{\parskip}{0.5\baselineskip}

\mdfdefinestyle{example}{
    skipbelow=0cm,
}
\newmdenv[style=example]{examplebox}

\EnableCrossrefs
\RecordChanges

\def\thispkg{\pkg{ordrel}}

\newenvironment{implnote}{\paragraph{Implementation considerations:}}{}

\begin{document}
  \DocInput{ordrel.dtx}
  \PrintChanges
\end{document}
%</driver>

%<*package>
\NeedsTeXFormat{LaTeX2e}[1999/12/01]
\ProvidesPackage{ordrel}[2020/09/18 1.0 Flexible command for typesetting binary ordering relations]
\RequirePackage{xparse}     % powerful command syntax
\RequirePackage{options}    % key-value options for commands
\RequirePackage{extarrows}  % extensible arrows
%</package>

% \fi
%
% \changes{1.0}{2020/09/18}{The first version}
%
% \providecommand*{\url}{\texttt}
% \title{The \thispkg\ package}
% \author{Alexander Vasilevsky (kalaider) \\ \url{a.kalaider@yandex.ru}}
%
% \maketitle



% \section{Introduction}
%
% The {\thispkg} package provides conveninent commands that
% together help to typeset various binary ordering relations
% written in a form $a \circ b$, where $\circ$ is a binary
% operation. Some examples include ${<}$, ${\le}$ with various
% subscripts (and even superscripts), labeled ${\to}$, etc.
%
% It may become quite tedious to work with lots of relations
% especially when some of them need to have embellishments.
% Fairly common situation is $\ordrel[->]{r}_i$ where the
% value of $i$ is highly context-dependent (and may often
% be absent). One then needs to either type something like
% |\myrelation{r$_i$}| each time the different value of $i$
% is required, or define two versions of |\myR|, one with
% additional argument representing the index and the other
% without it, just because there is no (out-of-the-box) way
% to carry the subscript inside the label, i.e. |\myR_i|
% is not the right way to get the desired result.
%
% But what if it would be possible? The simplest way to
% handle subscripts is the ``|e_|'' argument specifier provided
% by \pkg{xparse}. One may then define |\myR| as a universal
% thin wrapper around |\myrelation| capable of typesetting
% desired relation in both indexed and unindexed forms transparently:
%
% \begin{verbatim}
% \NewDocumentCommand{\myR}{e_}{%
%     \myrelation{r\IfValueT{#1}{_{#1}}}}
% \end{verbatim}
%
% {\thispkg} exercises the same idea but in a more generic
% and convenient way, providing an ability to easily generate
% new shortcut commands, each of which takes the same range
% of various customization options.
%
% With the help of {\thispkg} the above example may be
% simplified to just the following:
%
% \begin{verbatim}
% \NewOrdRel*{r}{\myR}
% \end{verbatim}
%
% with additional capability to specify formatting options,
% default values for indexes, etc.
%
% There are still lots of not-so-easy-to-solve small and
% annoying issues such as spacing. One needs to surround the
% entire command with braces when the relation is meant to be
% used standalone (i.e. in operator notation) as in the
% following example:
%
% \[ \ordrel*{hb} = (\ordrel*{po} \cup \ordrel*{sw})^+ \]
%
% {\thispkg} helps here a lot, providing starred and unstarred
% versions of relation commands.


% \section{Features}\label{sec:features}
%
% \begin{itemize}
%    \item Flexible configuration based on a rich set of options.
%    \item A number of predefined relation layout classes,
%          so it is quite simple to replace the relation symbol
%          by almost anything.
%    \item A number of predefined relation styles, so in most common
%          cases there is no need to customize individual options at all.
%    \item Almost any internal working horse of the command is replaceable
%          by means of options, that is, configurable on per-command basis.
%    \item Flexible spacing to simplify use of the relation symbol in
%          different contexts (as binary operator, in ``point-free''
%          standalone operator notation, etc.).
%    \item \cs{NewOrdRel(X)} commands provide a convenient way of
%          defining new \cs{ordrel}-like command with default option
%          values, label, subscript and superscript.
%    \item The \cs{SetupOrdRel} command allows for global configurarion
%          manipulation.
%    \item All state changing commands (\cs{NewOrdRel(X)}, \cs{SetupOrdRel})
%          are scoped, i.e. they have effect only inside their natural scope,
%          not relying on global definitions at all.
% \end{itemize}



% \section{Introductory example}\label{sec:example}
%
% What follows is a small text and the corresponding {\LaTeX}
% code that show a simple yet complete example of provided
% features. The first line of the example code is not strictly
% necessary but included for didactic purposes.
%
% \begin{examplebox}
%    \SetupOrdRel{->}
%    \NewOrdRel*{hb}{\hb}
%    \NewOrdRel*{po}{\po}
%
%    The \textit{happens-before} relation denoted as \hb\ (or
%    \hb[le] in some formal literature) is a partial order
%    relation defined as follows:
%
%    \[\hb* \ordrel[eq]{def} (\po* \cup \ordrel*{sw})^+,\]
%
%    where the \textit{program order} relation \po\ is a union
%    of all individual program orders $\po_i$ across all
%    threads of a particular \textit{execution}, and \ordrel{sw}
%    is a \textit{synchronizes-with} relation. We say ``$a$
%    \textit{happens-before} $b$'' and write $a \hb b$ if and
%    only if either of the following holds:
%
%    \begin{enumerate}
%        \item $a \po b$, or
%        \item $a \ordrel{sw} b$, or
%        \item there is some $x$, such that $a \hb x \hb b$.
%    \end{enumerate}
%
%    In some literature one may find variants of \hb, such as
%    \hb[label=ihb], \hb[label=ghb], etc. but they appear in
%    specialized contexts and are out of scope of our generic
%    discussion.
% \end{examplebox}
%
% \begin{examplebox}
% \begin{verbatim}
% \SetupOrdRel{->}
% \NewOrdRel*{hb}{\hb}
% \NewOrdRel*{po}{\po}
%
% The \textit{happens-before} relation denoted as \hb\ (or
% \hb[le] in some formal literature) is a partial order
% relation defined as follows:
%
% \[\hb* \ordrel[eq]{def} (\po* \cup \ordrel*{sw})^+,\]
%
% where the \textit{program order} relation \po\ is a union
% of all individual program orders $\po_i$ across all
% threads of a particular \textit{execution}, and \ordrel{sw}
% is a \textit{synchronizes-with} relation. We say ``$a$
% \textit{happens-before} $b$'' and write $a \hb b$ if and
% only if either of the following holds:
%
% \begin{enumerate}
%     \item $a \po b$, or
%     \item $a \ordrel{sw} b$, or
%     \item there is some $x$, such that $a \hb x \hb b$.
% \end{enumerate}
%
% In some literature one may find variants of \hb, such as
% \hb[label=ihb], \hb[label=ghb], etc. but they appear in
% specialized contexts and are out of scope of our generic
% discussion.
% \end{verbatim}
% \end{examplebox}



% \section{Usage}\label{sec:usage}
%
% \DescribeMacro{\ordrel}
% The main command provided by {\thispkg}. It takes options in
% a first argument, mandatory label argument and optional
% subscript and superscript for the label. Normally the command
% has \cs{mathrel}-like spacing but it may be changed to no
% spacing by means of options or by using starred version of
% the command.
%
% \begin{syntax}
%    \cs{ordrel}*\oarg{opts}\Arg{label}\meta{\_}\meta{subscript}\meta{\^}\meta{superscript}
%    \cs{ordrel}\oarg{opts}\Arg{label}\meta{\_}\meta{subscript}\meta{\^}\meta{superscript}
% \end{syntax}
%
% Starred version differs from normal version in that is sets up
% |spacing=no| option implicitly as the last option, so it always
% takes precedence. The main use case for such specialization is
% operator notation (such as in $\ordrel*[<]{r}_1 \cup \ordrel*[<]{r}_2$
% where relations $r_1$ and $r_2$ are used as arguments of another
% binary operator $\cup$).
%
% Note that subscript and superscript are applied \textbf{to label},
% not to the entire relation operator. If both subscript and superscript
% need to be specified, they must be given in the order described above.
%
% Commands defined by \cs{NewOrdRel(X)} have the same syntax and
% semantics except for starred versions of \cs{NewOrdRel(X)}
% which define \cs{ordrel}-like commands without the label argument.
%
%
% \DescribeMacro{\NewOrdRel}
% \DescribeMacro{\NewOrdRelX}
% Allows to define custom \cs{ordrel}-like command with a given set
% of default options and (optionally) explicitly provided label.
%
% \begin{syntax}
%    \cs{NewOrdRel}*\oarg{opts}\Arg{label}\Arg{command}
%    \cs{NewOrdRel}\oarg{opts}\Arg{command}
%    \cs{NewOrdRelX}*\oarg{opts}\Arg{label}\Arg{command}\meta{\_}\meta{subscript}\meta{\^}\meta{superscript}
%    \cs{NewOrdRelX}\oarg{opts}\Arg{command}\meta{\_}\meta{subscript}\meta{\^}\meta{superscript}
% \end{syntax}
%
% \cs{NewOrdRelX} is a convenient way of specifying default values for subscript
% and superscript of the command to be defined. It is roughly equivalent to
%
% \begin{verbatim}
% \NewOrdRel[subscript=<subscriptX>, superscript=<superscriptX>]...
% \end{verbatim}
%
% where option is provided only when the corresponding embellishment is
% specified for the \cs{NewOrdRelX} command.
%
% All starred versions of \cs{NewOrdRel(X)} commands effectively provide
% label value as an option to the command to be defined, so one may
% later change the label even if the newly defined command does not
% take it explicitly (see example below).
%
% \cs{NewOrdRel(X)} commands work on the logical scope level. Their behavior
% is similar to behavior of \cs{newcommand} in that trying to redefine
% already defined command will certainly fail.
%
% Here is an example of both starred and normal \cs{NewOrdRelX} usage:
%
% \begin{examplebox}
%     \NewOrdRelX[->]{\myarrowj}_j
%     \NewOrdRelX*[eq]{def}{\mystarreddef}^*
%
%     \myarrowj[it]{happens-before}
%     \myarrowj[it]{happens-before}_k
%     \mystarreddef[bf]
%     \mystarreddef[bf]^+
%     \mystarreddef[label=$\triangle$]^+
% \end{examplebox}
%
% \begin{examplebox}
% \begin{verbatim}
% \NewOrdRelX[->]{\myarrowj}_j
% \NewOrdRelX*[eq]{def}{\mystarreddef}^*
%
% \myarrowj[it]{happens-before}
% \myarrowj[it]{happens-before}_k
% \mystarreddef[bf]
% \mystarreddef[bf]^+
% \mystarreddef[label=$\triangle$]^+
% \end{verbatim}
% \end{examplebox}
%
%
% \DescribeMacro{\SetupOrdRel}
% Allows to change options globally. All commands in the current
% scope that don't override the specified options will use their
% new values.
%
% \begin{syntax}
%    \cs{SetupOrdRel}\Arg{opts}
%    \cs{SetupOrdRel}*\Arg{opts}
% \end{syntax}
%
% The normal command \textbf{always appends} provided option list
% to the state and never explicitly removes them, so calling it
% multiple times from the same scope may result in unnecessary
% overhead. To reduce the overhead, one may use starred version
% which \textbf{overwrites} the configuration instead of
% modifying it.
%
% One of the possible scenarios when this command may be helpful
% is global setup. One may configure {\thispkg} appropriately
% to the style of the document and provide further configuration
% as needed on a per-command basis.
%
% Here is how one may use \cs{SetupOrdRel}.
%
% \begin{examplebox}
%     \SetupOrdRel*{le,it,decorator=\text{\textcolor{red}{\ARG}}}
%     Red italic label: $a \ordrel{hb} b$ \\
%     {
%         \SetupOrdRel{->,bf}
%         Red boldface label, arrow symbol: $a \ordrel{sw} b$ \\
%     }
%     Red italic label: $a \ordrel{hb} b$
% \end{examplebox}
%
% \begin{examplebox}
% \begin{verbatim}
% \SetupOrdRel*{le,it,decorator=\text{\textcolor{red}{\ARG}}}
% Red italic label: $a \ordrel{hb} b$ \\
% {
%     \SetupOrdRel{->,bf}
%     Red boldface label, arrow symbol: $a \ordrel{sw} b$ \\
% }
% Red italic label: $a \ordrel{hb} b$
% \end{verbatim}
% \end{examplebox}


% \section{Options}\label{sec:options}
%
% All \cs{ordrel}-like commands take the same options.
%
% \DescribeMacro{layoutclass=\meta{xcmd, cmd, subscript, superscript}}
% \DescribeMacro{relationsymbol=\meta{cmd1, hbox}}
% The |layoutclass| option determines relation symbol layout and
% places requirements on what can be assigned to the |relationsymbol|
% option:
%
% \begin{itemize}
%     \item With |xcmd| the |relationsymbol| command is considered
%           \textit{eXtensible}, i.e. width of the relation symbol
%           is expected to changes appropriately to accommodate
%           the entire label. Label padding \textbf{is} applied.
%           |relationsymbol| should take exactly one argument~---
%           fully formatted label. Examples: \cs{xrightarrow}.
%     \item With |cmd| the |relationsymbol| command is considered
%           fixed-width. Adding a padding to the label changes
%           nothing in such case, so it is \textbf{not} applied.
%           |relationsymbol| should take exactly one argument~---
%           fully formatted label. Examples: |\overset{#1}{symb}|,
%           where |symb| is \cs{to}, \cs{circ}, built-in $=$, etc.
%     \item With |subscript| and |superscript| the |relationsymbol|
%           command is considered constant, i.e. it should not take
%           any arguments. The |layoutcmd| manually places the label
%           either in subscript or superscript position depending
%           on the value of |layoutclass|. Label padding is
%           \textbf{not} applied. Use the |decorator| option if
%           you need to visually adjust the symbol and you don't
%           want to provide your own formatting command to
%           |relationsymbol| setting |layoutclass| to |cmd|. Examples:
%           built-in $<$, $>$, $=$, \cs{circ}, A, etc.
% \end{itemize}
%
% |relationsymbol| is expanded inside math mode.
%
% The following example demonstrates the use of |layoutclass|
% and |relationsymbol| together.
%
% \begin{examplebox}
%     \def\myextarrow#1{\xrightarrow[#1]{}}
%     \def\mynonextdef#1{\stackrel{#1}{\equiv}}
%     \def\myrelsymbol{\sqsupseteq}
%
%     \SetupOrdRel*{
%         layoutclass = xcmd,
%         relationsymbol = \myextarrow,
%     }
%     \ordrel{under}
%     \SetupOrdRel*{
%         layoutclass = cmd,
%         relationsymbol = \mynonextdef,
%     }
%     \ordrel{$\triangle$}
%     \SetupOrdRel*{
%         layoutclass = subscript,
%         relationsymbol = \myrelsymbol,
%     }
%     \ordrel{square}
% \end{examplebox}
%
% \begin{examplebox}
% \begin{verbatim}
% \def\myextarrow#1{\xrightarrow[#1]{}}
% \def\mynonextdef#1{\stackrel{#1}{\equiv}}
% \def\myrelsymbol{\sqsupseteq}
%
% \SetupOrdRel*{
%     layoutclass = xcmd,
%     relationsymbol = \myextarrow,
% }
% \ordrel{under}
% \SetupOrdRel*{
%     layoutclass = cmd,
%     relationsymbol = \mynonextdef,
% }
% \ordrel{$\triangle$}
% \SetupOrdRel*{
%     layoutclass = subscript,
%     relationsymbol = \myrelsymbol,
% }
% \ordrel{square}
% \end{verbatim}
% \end{examplebox}
%
% \DescribeMacro{spacing=\meta{no, rel}}
% Determines whether the entire relation command should
% behave as a relation operator in math formulas. If
% set to |rel|, the enire command is wrapped into \cs{mathrel}.
%
% Note that if starred version of \cs{ordrel}-like command
% is used, this option is implicitly set to |no|. It is
% by far the most convenient way of saying |spacing=no|
% for a particular command.
%
% The following example provides comparison of different
% spacing modes in two orthogonal contexts.
%
% \begin{examplebox}
%     Relation as binary operation, |spacing=rel|: $a\ordrel{hb}b$              \\
%     Relation as binary operation, |spacing=no|: $a\ordrel*{hb}b$              \\
%     Relation as operator, |spacing=rel|: $\ordrel{po}\subseteq\ordrel{hb}$    \\
%     Relation as operator, |spacing=no|: $\ordrel*{po}\subseteq\ordrel*{hb}$   \\
% \end{examplebox}
%
% \begin{examplebox}
% \begin{verbatim}
% Relation as binary operation, |spacing=rel|: $a\ordrel{hb}b$              \\
% Relation as binary operation, |spacing=no|: $a\ordrel*{hb}b$              \\
% Relation as operator, |spacing=rel|: $\ordrel{po}\subseteq\ordrel{hb}$    \\
% Relation as operator, |spacing=no|: $\ordrel*{po}\subseteq\ordrel*{hb}$   \\
% \end{verbatim}
% \end{examplebox}
%
% \DescribeMacro{labelpadding=\meta{length}}
% Determines the amount of horizontal padding that should be
% added to fully formatted label before passing it to
% |relationsymbol| in case of |layoutclass=xcmd|. Otherwise
% is ignored.
%
% It may be convenient to add some padding to extensible arrow labels.
% {\thispkg} by default defines this option non-zero. The following
% example demonstrates the effect of setting the option to other value.
%
% \begin{examplebox}
%     \def\mydef#1{\stackrel{#1}{\equiv}}
%     Default padding: $a\ordrel{hb}b$                              \\
%     No padding: $a\ordrel[labelpadding=0pt]{hb}b$                 \\
%     Negative padding: $a\ordrel[labelpadding=-0.28em]{hb}b$       \\
%     Irrelevant: $a\ordrel[le,labelpadding=3em]{hb}b$              \\
%     Irrelevant: $a\ordrel[nxsymbol=\mydef,labelpadding=3em]{hb}b$ \\
% \end{examplebox}
%
% \begin{examplebox}
% \begin{verbatim}
% \def\mydef#1{\stackrel{#1}{\equiv}}
% Default padding: $a\ordrel{hb}b$                              \\
% No padding: $a\ordrel[labelpadding=0pt]{hb}b$                 \\
% Negative padding: $a\ordrel[labelpadding=-0.28em]{hb}b$       \\
% Irrelevant: $a\ordrel[le,labelpadding=3em]{hb}b$              \\
% Irrelevant: $a\ordrel[nxsymbol=\mydef,labelpadding=3em]{hb}b$ \\
% \end{verbatim}
% \end{examplebox}
%
% \DescribeMacro{decorator=\meta{argcmd}}
% \DescribeMacro{labeldec=\meta{argcmd}}
% \DescribeMacro{subdec=\meta{argcmd}}
% \DescribeMacro{supdec=\meta{argcmd}}
% Decorator to be applied to the specified part of the label.
%
% \textit{Decorator} is a command which takes its argument
% as an \cs{ARG} macro. It is a convenient way of putting
% complex decorations on \cs{ARG} without the need to even
% define any command.
%
% All decorators are expanded inside math mode.
%
% |decorator| affects the entire label, i.e. its \cs{ARG}
% is a fully formatted (but not yet decorated) label.
% Its default value is no-op, i.e. just |\ARG|.
%
% All other decorators decorate their specific parts of
% a fully formatted label, specifically, the label itself,
% subscript and superscript. All of them are no-op by default,
% except for |labeldec| which is |\textnormal{\ARG}|.
%
% The following example demonstrates the effect of providing
% all four possible decorators.
%
% \begin{examplebox}
%     \SetupOrdRel{
%         decorator = \displaystyle\colorbox{black!30}{\ARG},
%         labeldec  = \displaystyle\textnormal{\textcolor{red!80!black}{\ARG}},
%         subdec    = \displaystyle\textcolor{green!40!black}{\mathbf{\ARG}},
%         supdec    = \displaystyle\textcolor{blue!80!black}{\overline{\ARG}},
%     }
%     \Large$\ordrel[->]{relation}_{sub}^{sup}$
% \end{examplebox}
%
% \begin{examplebox}
% \begin{verbatim}
% \SetupOrdRel{
%     decorator = \displaystyle\colorbox{black!30}{\ARG},
%     labeldec  = \displaystyle\textnormal{\textcolor{red!80!black}{\ARG}},
%     subdec    = \displaystyle\textcolor{green!40!black}{\mathbf{\ARG}},
%     supdec    = \displaystyle\textcolor{blue!80!black}{\overline{\ARG}},
% }
% \Large$\ordrel[->]{relation}_{sub}^{sup}$
% \end{verbatim}
% \end{examplebox}
%
% \DescribeMacro{label=\meta{hbox}}
% \DescribeMacro{subscript=\meta{hbox}}
% \DescribeMacro{superscript=\meta{hbox}}
% Default values for different parts of the fully formatted label.
%
% |label| makes sense only for commands defined by starred
% versions of \cs{NewOrdRel(X)} that don't take label explicitly.
%
% The example follows.
%
% \begin{examplebox}
%     \SetupOrdRel{
%         subscript = i,
%         superscript = +,
%     }
%     \Large
%     $\ordrel{relation}$
%     $\ordrel{relation}_k$
%     $\ordrel{relation}^*$
%     $\ordrel{relation}_k^*$
% \end{examplebox}
%
% \begin{examplebox}
% \begin{verbatim}
% \SetupOrdRel{
%     subscript = i,
%     superscript = +,
% }
% \Large
% $\ordrel{relation}$
% $\ordrel{relation}_k$
% $\ordrel{relation}^*$
% $\ordrel{relation}_k^*$
% \end{verbatim}
% \end{examplebox}
%
% \DescribeMacro{xsymbol=\meta{cmd1}}
% \DescribeMacro{nxsymbol=\meta{cmd1}}
% \DescribeMacro{symbol=\meta{hbox}}
% Convenient wrappers around |relationsymbol| that set
% |layoutclass| appropriately (|layoutclass=xcmd|,
% |cmd| and |subscript| correspondingly).
%
% \DescribeMacro{->, <-, <->}
% Convenient styles setting |xsymbol| to the appripriate
% extensible arrow command.
%
% \DescribeMacro{eq}
% Convenient style setting |xsymbol| to \cs{xlongequal}.
%
% \DescribeMacro{>, <, le, ge}
% Convenient styles setting |symbol| to the appripriate
% relation symbol.
%
% \DescribeMacro{raw}
% Same as |spacing=no|.
%
% \DescribeMacro{text, it, bf}
% Convenient styles setting |labeldec| to the appropriate
% text-mode style.
%
% \DescribeMacro{math}
% Resets |labeldec| to no-op, effectively leaving the label
% inside math mode.


% \section{Advanced options}\label{sec:avdanced-options}
%
% \DescribeMacro{layoutcmd=\meta{cmd1}}
% \DescribeMacro{spacingcmd=\meta{cmd1}}
% \DescribeMacro{labelpaddingcmd=\meta{cmd1}}
% \DescribeMacro{labelcmd=\meta{cmd1}}
% Low-level customization options. See the documentation of
% their default values for more details on how to implement
% them and for the reasons one may need it at all.
%
% All these commands by default run inside math mode. The user
% should preserve this convention.
%
% \begin{examplebox}
%     \def\unpad#1{\!\!#1\!\!}
%
%     \SetupOrdRel*{eq}
%     \ordrel[labelpadding=-0.28em]{$\triangle$}
%     \ordrel[labelpaddingcmd=\unpad]{$\triangle$}
%
%     \def\labelcmd#1#2#3{{}\IfValueT{#2}{_{#2}}\IfValueT{#3}{^{#3}}#1}
%
%     \makeatletter
%         \SetupOrdRel*{labelcmd=\@ordrel@labelcmd@trampoline{\labelcmd}}
%     \makeatother
%     \ordrel{label}_{sub}^{sup}
%
%     \def\layoutcmd#1{\fbox{#1}}
%
%     \SetupOrdRel*{<}
%     \ordrel[layoutcmd=\layoutcmd]{label}_{sub}^{sup}
%     \ordrel[nxsymbol=\fbox]{label}_{sub}^{sup}
% \end{examplebox}
%
% \begin{examplebox}
% \begin{verbatim}
% \def\unpad#1{\!\!#1\!\!}
%
% \SetupOrdRel*{eq}
% \ordrel{$\triangle$}
% \ordrel[labelpadding=-0.28em]{$\triangle$}
% \ordrel[labelpaddingcmd=\unpad]{$\triangle$}
%
% \def\labelcmd#1#2#3{{}\IfValueT{#2}{_{#2}}\IfValueT{#3}{^{#3}}#1}
%
% \makeatletter
%     \SetupOrdRel*{labelcmd=\@ordrel@labelcmd@trampoline{\labelcmd}}
% \makeatother
% \ordrel{label}_{sub}^{sup}
%
% \def\layoutcmd#1{\fbox{#1}}
%
% \SetupOrdRel*{<}
% \ordrel[layoutcmd=\layoutcmd]{label}_{sub}^{sup}
% \ordrel[nxsymbol=\fbox]{label}_{sub}^{sup}
% \end{verbatim}
% \end{examplebox}
%
% In the above example only one case (currently) cannot be achieved
% by using high-level options.


% \section{User-provided styles and options}\label{sec:custom-options}
%
% As {\thispkg} internally uses the \pkg{options} package to
% handle options, there are planty ways to extend a set of
% available options to provide some additional functionality.
% The easiest and by far the most common case is a custom
% style definition. The more interesting case is definition
% of a completely unrelated option consumed by relation symbol
% command or decorator. What follows is a list of examples
% of the actual implementation of the above two cases.
%
% \begin{examplebox}
%     \options{
%         /ordrel/dbl->/.new style* = { /ordrel/xsymbol = \xLongrightarrow },
%         /mystyles/.new family,
%         /mystyles/<-dbl/.new style* = { /ordrel/xsymbol = \xLongleftarrow },
%     }
%     \SetupOrdRel{
%         >>/.new style* = { symbol = {>>} },
%         /ordrel/<</.new style* = { /ordrel/symbol = {>>} },
%         labelcolor/.new color = black,
%         labeldec = \textnormal{\textcolor{\option{/ordrel/labelcolor}}{\ARG}},
%     }
%     Using relative option paths: \ordrel[>>]{relation}, \\
%     Using absolute option paths: \ordrel[<<]{relation}, \\
%     Using plain \pkg{options}: \ordrel[dbl->]{relation}, \\
%     Using custom options family: \ordrel[/mystyles/<-dbl]{relation}, \\
%     Consuming custom option in decorator: \ordrel[labelcolor=green]{relation}, \\
% \end{examplebox}
%
% \begin{examplebox}
% \begin{verbatim}
% \options{
%     /ordrel/dbl->/.new style* = { /ordrel/xsymbol = \xLongrightarrow },
%     /mystyles/.new family,
%     /mystyles/<-dbl/.new style* = { /ordrel/xsymbol = \xLongleftarrow },
% }
% \SetupOrdRel{
%     >>/.new style* = { symbol = {>>} },
%     /ordrel/<</.new style* = { /ordrel/symbol = {>>} },
%     labelcolor/.new color = black,
%     labeldec = \textnormal{\textcolor{\option{/ordrel/labelcolor}}{\ARG}},
% }
% Using relative option paths: \ordrel[>>]{relation}, \\
% Using absolute option paths: \ordrel[<<]{relation}, \\
% Using plain \pkg{options}: \ordrel[dbl->]{relation}, \\
% Using custom options family: \ordrel[/mystyles/<-dbl]{relation}, \\
% Consuming custom option in decorator: \ordrel[labelcolor=green]{relation}, \\
% \end{verbatim}
% \end{examplebox}

% \section{Limitations}\label{sec:limitations}
%
% The package requires the catcodes of ``\_'' and ``\^{}'' to be
% consistent inside and outside of the document body for its
% full and consistent operation. See \cref{sec:troubleshooting}
% for details and possible fixes.



% \section{Troubleshooting}\label{sec:troubleshooting}
%
% \subsection{\pkg{underscore}}
%
% One of the notable features \thispkg\ provides is transparent
% handling of subscript and superscript embellishments. One
% need not to tediously type something like the following:
% \begin{verbatim}
% \def\hbi{\ordrel{hb$_i$}}
% \def\hbj{\ordrel{hb$_j$}}
% \def\hbk{\ordrel{hb$_k$}}
% ... where \hbi is ..., \hbj is ... and \hbk is ...
% \end{verbatim}
% while one's intension is completely clear and fairly common:
% \begin{verbatim}
% \def\hb{\ordrel{hb}}
% ... where \hb_i is ..., \hb_j is ... and \hb_k is ...
% \end{verbatim}
%
% This feature, however, doesn't come without a cost. {\thispkg}
% relies on \pkg{xparse} internally, which, in turn, expects
% the catcode of ``\_'' (and ``\^{}'') not to change suddenly.
% The \pkg{underscore} package changes catcode of ``\_'' at the
% begin of the document, so that commands defined outside of
% the document body might not work correcly inside it.
%
% The workaround here is to pass the |strings| option to \pkg{underscore}.
% It, however, may be destructive to unconditionally set this option.
% Moreover, there is no sense in passing options to already loaded package.
% So {\thispkg} is configured to only emit a warning if it detects
% \pkg{underscore}. The user should either pass |strings| to \pkg{underscore}
% manually (and optionally turn off the warning by passing |quiet| to
% {\thispkg}) or (if \pkg{underscore} is loaded after {\thispkg})
% pass |underscore| to {\thispkg} to let {\thispkg} configure
% \pkg{underscore} automatically.


% \StopEventually{}
%
% \section{Implementation}
%
% \iffalse
%<*package>
% \fi

% \subsection{Package Options}
%
% The package comes without global options affecting provided
% commands (except for the following to ensure better
% interoperability with interfering packages). All provided
% features are configurable in-place on per-command or per-scope basis.
%
% Don't emit warning if \pkg{underscore} is loaded before
% {\thispkg}, or if \pkg{underscore} is loaded after {\thispkg}
% and the |underscore| option is not provided, or if some other
% package changes catcodes of ``\_'' and/or ``\^{}'' inside
% the document body
%    \begin{macrocode}
\newif\if@ordrel@opt@quiet
\DeclareOption{quiet}{\@ordrel@opt@quiettrue}
%    \end{macrocode}
% Whether to pass the |strings| option to \pkg{underscore}
% during {\thispkg} initialization
%    \begin{macrocode}
\newif\if@ordrel@opt@underscore@strings
\DeclareOption{underscore}{\@ordrel@opt@underscore@stringstrue}
%    \end{macrocode}
% Handle unknown options
%    \begin{macrocode}
\DeclareOption*{%
    \PackageWarning{ordrel}{`\CurrentOption' ignored}}
%    \end{macrocode}
% Some reasonable defaults
%    \begin{macrocode}
\ExecuteOptions{}
%    \end{macrocode}
% Done with package options
%    \begin{macrocode}
\ProcessOptions\relax
%    \end{macrocode}

% \subsection{Interaction with other packages}
%
% Emit warning if \pkg{underscore} is loaded before {\thispkg},
% otherwise configure \pkg{underscore} appropriately. Warn if someone
% else plays with catcodes in a way that it may interfere with us.
%    \begin{macrocode}
\newif\if@ordrel@catcode@warned
\if@ordrel@opt@quiet\else

    \@ifpackageloaded{underscore}%
    {%
        \@ordrel@catcode@warnedtrue%
        \PackageWarning{ordrel}{%
            The `underscore' package was loaded before `ordrel'. %
            It may cause some commands to handle subscript embellishments %
            improperly. Please, load `underscore' after `ordrel' %
            and use `underscore' option to overcome this, or add the %
            `quiet' option and pass `strings' to `underscore' manually.%
        }
    }{}

    \if@ordrel@opt@underscore@strings\relax
        \PassOptionsToPackage{strings}{underscore}
    \fi

    \if@ordrel@catcode@warned\else
        \AtBeginDocument{%
            \@ifpackageloaded{underscore}%
            {%
                \if@ordrel@opt@underscore@strings\else
                    \PackageWarning{ordrel}{%
                        The `underscore' package was loaded after `ordrel' while %
                        `ordrel' was not instructed to properly configure the former. %
                        It may cause some commands to handle subscript embellishments %
                        improperly. Please, provide `underscore' option to configure %
                        the `underscore' package automatically, or `quiet' to %
                        suppress this warning if you prefer to manually pass `strings' %
                        to the `underscore' package or just don't care.%
                    }
                \fi
            }{%
                \begingroup
                \def\emitwarning{%
                    \PackageWarning{ordrel}{%
                        Catcodes of `_' and/or `^' characters are non-standard. %
                        It may cause some `ordrel` commands to handle subscript %
                        embellishments improperly. If non-standard catcodes are %
                        expected and the package works fine, suppress this warning %
                        with the help of the `quiet' option.%
                    }
                }
                \ifnum\the\catcode`\^=7\relax\else
                    \@ordrel@catcode@warnedtrue
                    \emitwarning
                \fi
                \ifnum\the\catcode`\_=8\relax\else
                    \if@ordrel@catcode@warned\else
                        \emitwarning
                    \fi
                \fi
                \endgroup
            }
        }
    \fi
\fi
%    \end{macrocode}

% \subsection{Command Options}
%
%    \begin{macrocode}
\options{
    % options
    /ordrel/.new family,
    /ordrel/layoutcmd/.new value        = \@ordrel@layoutcmd,
    /ordrel/layoutclass/.new choice     = { xcmd, cmd, subscript, superscript },
    /ordrel/relationsymbol/.new value,
    /ordrel/spacingcmd/.new value       = \@ordrel@spacingcmd,
    /ordrel/spacing/.new choice         = { no, rel },
    /ordrel/spacing                     = rel,
    /ordrel/labelpaddingcmd/.new value  = \@ordrel@labelpaddingcmd,
    /ordrel/labelpadding/.new value     = 0.2em,
    /ordrel/labelcmd/.new value         = \@ordrel@labelcmd@trampoline{\@ordrel@labelcmd},
    /ordrel/decorator/.new value        = \ARG,
    /ordrel/labeldec/.new value         = \textnormal{\ARG},
    /ordrel/subdec/.new value           = \ARG,
    /ordrel/supdec/.new value           = \ARG,
    % label and script defaults
    /ordrel/label/.new value            =,
    /ordrel/subscript/.new value        =,
    /ordrel/superscript/.new value      =,
    % styles
    /ordrel/xsymbol/.new style = {
        /ordrel/layoutclass = xcmd,
        /ordrel/relationsymbol = #1,
    },
    /ordrel/nxsymbol/.new style = {
        /ordrel/layoutclass = cmd,
        /ordrel/relationsymbol = #1,
    },
    /ordrel/symbol/.new style = {
        /ordrel/layoutclass = subscript,
        /ordrel/relationsymbol = #1,
    },
    /ordrel/->/.new style* = {
        /ordrel/xsymbol = \xrightarrow,
    },
    /ordrel/<-/.new style* = {
        /ordrel/xsymbol = \xleftarrow,
    },
    /ordrel/<->/.new style* = {
        /ordrel/xsymbol = \xleftrightarrow,
    },
    /ordrel/eq/.new style* = {
        /ordrel/xsymbol = \xlongequal,
        /ordrel/labelpadding = -0.28em,
    },
    /ordrel/>/.new style* = {
        /ordrel/symbol = >,
    },
    /ordrel/</.new style* = {
        /ordrel/symbol = <,
    },
    /ordrel/le/.new style* = {
        /ordrel/symbol = \le,
    },
    /ordrel/ge/.new style* = {
        /ordrel/symbol = \ge,
    },
    /ordrel/raw/.new style* = {
        /ordrel/spacing = no,
    },
    /ordrel/math/.new style* = {
        /ordrel/labeldec = \ARG,
    },
    /ordrel/text/.new style* = {
        /ordrel/labeldec = \textnormal{\ARG},
    },
    /ordrel/it/.new style* = {
        /ordrel/labeldec = \textnormal{\textit{\ARG}},
    },
    /ordrel/bf/.new style* = {
        /ordrel/labeldec = \textnormal{\textbf{\ARG}},
    },
    /ordrel/->,
}

% global options setup
\def\@ordrel@options@global{}
%    \end{macrocode}

% \subsection{Internal formatters}

% \begin{macro}{\@ordrel@labelpaddingcmd}
%    Given a fully formatted label box, returns its padded variant.
%
%    \begin{arguments}
%        \item fully formatted label box with all subscripts
%              and superscripts already applied
%    \end{arguments}
%
%    \begin{implnote}
%       The command should but not required to take the |/ordrel/labelpadding|
%       option into account. It may provide its own padding strategy if necessary.
%    \end{implnote}
%
%    \begin{macrocode}
\DeclareDocumentCommand{\@ordrel@labelpaddingcmd}{m}{%
    \letoption{/ordrel/labelpadding}\@ordrel@labelpadding\relax%
    \hspace*{\@ordrel@labelpadding}%
    #1%
    \hspace*{\@ordrel@labelpadding}%
}
%    \end{macrocode}
% \end{macro}

% \begin{macro}{\@ordrel@layoutcmd}
%    Given a fully formatted label box, returns the full relation
%    consisting of the relation symbol itself and the label,
%    padded if necessary.
%
%    \begin{arguments}
%        \item fully formatted label box with all subscripts
%              and superscripts already applied
%    \end{arguments}
%
%    \begin{implnote}
%       The command should but not required to take |/ordrel/layoutclass|,
%       |/ordrel/relationsymbol| and |/ordrel/labelpaddingcmd| options
%       into account. It may provide its own symbol and may even completely
%       ignore the argument (label) or format its own one from some custom options.
%    \end{implnote}
%
%    \begin{macrocode}
\DeclareDocumentCommand{\@ordrel@layoutcmd}{m}{%
    \letoption{/ordrel/layoutclass/@ord}\@ordrel@layoutclass\relax%
    \letoption{/ordrel/relationsymbol}\@ordrel@relationsymbol\relax%
    \@ordrel@util@switch@four{\@ordrel@layoutclass}%
        {\@ordrel@relationsymbol{\option{/ordrel/labelpaddingcmd}{#1}}}% xcmd
        {\@ordrel@relationsymbol{#1}}% cmd
        {{\@ordrel@relationsymbol}_{#1}}% subscript
        {{\@ordrel@relationsymbol}^{#1}}% superscript
        {\PackgeError{ordrel}{Unknown arrow type option: `\option{/ordrel/layoutclass}'}}%
}
%    \end{macrocode}
% \end{macro}

% \begin{macro}{\@ordrel@spacingcmd}
%    Given a fully formatted relation box, applies the specified
%    (implicitly by means of options) overall spacing policy.
%
%    \begin{arguments}
%        \item fully formatted relation box
%    \end{arguments}
%
%    \begin{implnote}
%       The command should but not required to take the |/ordrel/spacing|
%       option into account. Currently only binary relation spacing and
%       no spacing alternatives are provided but in some exotic cases
%       it may be desireable to achieve e.g. operator-like spacing.
%    \end{implnote}
%
%    \begin{macrocode}
\DeclareDocumentCommand{\@ordrel@spacingcmd}{m}{%
    \letoption{/ordrel/spacing/@ord}\@ordrel@spacing\relax%
    \@ordrel@util@switch@two{\@ordrel@spacing}%
        {{#1}}% no
        {\mathrel{{#1}}}% rel
        {\PackgeError{ordrel}{Unknown arrow type option: `\option{/ordrel/spacing}'}}%
}
%    \end{macrocode}
% \end{macro}

% \begin{macro}{\@ordrel@labelcmd@trampoline}
%    The helper function that invokes the simplier |labelcmd| with
%    some arguments defaulted to their corresponding option values
%    or \cs{NoValue} if the option is itself void.
%
%    \begin{arguments}
%        \item the command to be invoked
%        \item raw main label (may be \cs{NoValue})
%        \item raw subscript (may be \cs{NoValue})
%        \item raw superscript (may be \cs{NoValue})
%    \end{arguments}
%
%    \begin{implnote}
%       The command should but not required to take the |/ordrel/spacing|
%       option into account. Currently only binary relation spacing and
%       no spacing alternatives are provided but in some exotic cases
%       it may be desireable to achieve e.g. operator-like spacing.
%    \end{implnote}
%
%    \begin{macrocode}
\DeclareDocumentCommand{\@ordrel@labelcmd@trampoline}{m m m m}{%
    \@ordrel@util@invoke@with@default{#4}{/ordrel/superscript}{%
        \@ordrel@util@invoke@with@default{#3}{/ordrel/subscript}{%
            \@ordrel@util@invoke@with@default{#2}{/ordrel/label}{#1}
        }%
    }%
}
%    \end{macrocode}
% \end{macro}

% \begin{macro}{\@ordrel@labelcmd}
%    Combines label text and two scripts into a single label box.
%
%    \begin{arguments}
%        \item main label (may be \cs{NoValue})
%        \item subscript (may be \cs{NoValue})
%        \item superscript (may be \cs{NoValue})
%    \end{arguments}
%
%    \begin{implnote}
%       The command should but not required to take the following options into account:
%
%       \begin{itemize}
%           \item defaults for each argument:
%                 |/ordrel/label|, |/ordrel/subscript| and |/ordrel/superscript|,
%           \item entire label decorators:
%                 |/ordrel/decorator|.
%           \item decorators for each part of the label:
%                 |/ordrel/labeldec|, |/ordrel/subdec| and |/ordrel/supdec|.
%       \end{itemize}
%
%       All decorators should but not required to be expanded inside math mode.
%
%       In order to simplify implementation, the package actually installs
%       a thin wrapper around this macro as a |labelcmd| that prepares options
%       from the first bullet above as regular arguments. The typical implementation
%       need not handle arguments and options separately.
%    \end{implnote}
%
%    \begin{macrocode}
\DeclareDocumentCommand{\@ordrel@labelcmd}{m m m}{%
    \@ordrel@util@decorator@run{\option{/ordrel/decorator}}{%
        \IfValueT{#1}{\@ordrel@util@decorator@run{\option{/ordrel/labeldec}}{#1}}%
        % in case if the outer decorator enters text mode,
        % we still should be able to use _ and ^
        \ensuremath{%
            \IfValueT{#2}{_\@ordrel@util@decorator@run{\option{/ordrel/subdec}}{#2}}%
            \IfValueT{#3}{^\@ordrel@util@decorator@run{\option{/ordrel/supdec}}{#3}}%
        }%
    }%
}
%    \end{macrocode}
% \end{macro}

% \subsection{User-exposed command trampolines}

% \begin{macro}{\@ordrel@impl}
%    Combines label text and two scripts into a single relation box.
%
%    \begin{arguments}
%        \item main label (may be \cs{NoValue})
%        \item subscript (may be \cs{NoValue})
%        \item superscript (may be \cs{NoValue})
%    \end{arguments}
%
%    \begin{implnote}
%       The command should but not required to invoke the following commands
%       in order to fully format the result:
%
%       \begin{itemize}
%           \item |/ordrel/labelcmd|,
%           \item |/ordrel/layoutcmd|, and
%           \item |/ordrel/spacingcmd|.
%       \end{itemize}
%
%       The default implementation just chains them in that order.
%    \end{implnote}
%
%    \begin{macrocode}
\DeclareDocumentCommand{\@ordrel@impl}{m m m}{%
    \def\@ordrel@symbol{%
        \option{/ordrel/layoutcmd}{\option{/ordrel/labelcmd}{#1}{#2}{#3}}}%
    \option{/ordrel/spacingcmd}{\@ordrel@symbol}%
}
%    \end{macrocode}
% \end{macro}

% \begin{macro}{\@ordrel@new@label}
% \begin{macro}{\@ordrel@new@nolabel}
%    Define labeled or unlabeled variant of the new \cs{ordrel}-like command.
%
%    \begin{arguments}
%        \item (for internal needs) must NOT be provided with value
%        \item default option values for the new command
%        \item (in case of named) predefined label (cannot be changed)
%        \item command to define
%    \end{arguments}
%
%    \begin{macrocode}
\DeclareDocumentCommand{\@ordrel@new@label}{o m m m}{%
    \NewDocumentCommand{#4}{s O{} e_ e^}{%
        \begingroup%
            % local options take precedence
            \@ordrel@util@options{#2,label=#3,##2}{##1}%
            % take name from the outer command,
            % embellishments - from the inner
            \ensuremath{\@ordrel@impl{#1}{##3}{##4}}%
        \endgroup%
    }%
}

\DeclareDocumentCommand{\@ordrel@new@nolabel}{m m}{%
    \NewDocumentCommand{#2}{s O{} m e_ e^}{%
        \begingroup%
            % local options take precedence
            \@ordrel@util@options{#1,##2}{##1}%
            % take all arguments from the inner
            \ensuremath{\@ordrel@impl{##3}{##4}{##5}}%
        \endgroup%
    }%
}
%    \end{macrocode}
% \end{macro}
% \end{macro}


% \begin{macro}{\@ordrel@new@label@x}
% \begin{macro}{\@ordrel@new@nolabel@x}
%    Define labeled or unlabeled variant of the new \cs{ordrel}-like command.
%    Both commands are capable of handling embellishments which become
%    default values for subscript and superscript in the newly defined command.
%
%    \begin{arguments}
%        \item (for internal needs) must NOT be provided with value
%        \item default option values for the new command
%        \item (in case of named) predefined label (cannot be changed)
%        \item command to define
%        \item default value for subscript
%        \item default value for superscript
%    \end{arguments}
%
%    \begin{macrocode}
\DeclareDocumentCommand{\@ordrel@new@label@x}{o m m m e_ e^}{%
    % #1 must always be -NoValue-
    \NewDocumentCommand{#4}{s O{} e_ e^}{%
        \begingroup%
            % local options take precedence
            \@ordrel@util@options@x{#2,label=#3}{##2}{#5}{#6}{##1}
            % take name from the outer command,
            % local embellishments take precedence
            \ensuremath{\@ordrel@impl{#1}{##3}{##4}}%
        \endgroup%
    }%
}

\DeclareDocumentCommand{\@ordrel@new@nolabel@x}{m m e_ e^}{%
    \NewDocumentCommand{#2}{s O{} m e_ e^}{%
        \begingroup%
            % local options take precedence
            \@ordrel@util@options@x{#1}{##2}{#3}{#4}{##1}%
            % take name from the inner command,
            % local embellishments take precedence
            \ensuremath{\@ordrel@impl{##3}{##4}{##5}}%
        \endgroup%
    }%
}
%    \end{macrocode}
% \end{macro}
% \end{macro}

% \subsection{Various utility commands}

% \begin{macro}{\@ordrel@util@options}
%    Sets up options taking spacing issues into account.
%
%    \begin{arguments}
%        \item additional options
%        \item no spacing flag
%    \end{arguments}
%
%    \begin{macrocode}
\DeclareDocumentCommand{\@ordrel@util@options}{m m}{%
    \options{/ordrel,}%
    \@ordrel@options@global%
    \optionsalso{#1,}%
    \IfBooleanT{#2}{\optionsalso{spacing=no}}%
}
%    \end{macrocode}
% \end{macro}

% \begin{macro}{\@ordrel@util@options@x}
%    Sets up options taking care of embellishments
%    if provided and spacing as well.
%
%    \begin{arguments}
%        \item additional global options
%        \item additional local options
%        \item subscript (may be \cs{NoValue})
%        \item superscript (may be \cs{NoValue})
%        \item no spacing flag
%    \end{arguments}
%
%    \begin{macrocode}
\DeclareDocumentCommand{\@ordrel@util@options@x}{m m m m m}{%
    % it is simplier to duplicate some code
    % from the command above than call it
    \options{/ordrel,}%
    \@ordrel@options@global%
    \optionsalso{#1,}%
    \IfValueT{#3}{\optionsalso{subscript=#3}}%
    \IfValueT{#4}{\optionsalso{superscript=#4}}%
    \optionsalso{#2}%
    \IfBooleanT{#5}{\optionsalso{spacing=no}}%
}
%    \end{macrocode}
% \end{macro}

% \begin{macro}{\@ordrel@util@switch@two}
% \begin{macro}{\@ordrel@util@switch@three}
% \begin{macro}{\@ordrel@util@switch@four}
%    Index-based switch with the first argument as the key
%    and the last argument as the default case. Zero-based indexing.
%
%    \begin{macrocode}
\DeclareDocumentCommand{\@ordrel@util@switch@two}{m m m m}{%
    \ifnum#1=0\relax#2\else%
        \ifnum#1=1\relax#3\else#4\fi%
    \fi%
}
\DeclareDocumentCommand{\@ordrel@util@switch@three}{m m m m m}{%
    \ifnum#1=0\relax#2\else%
        \ifnum#1=1\relax#3\else%
            \ifnum#1=2\relax#4\else#5\fi%
        \fi%
    \fi%
}
\DeclareDocumentCommand{\@ordrel@util@switch@four}{m m m m m m}{%
    \ifnum#1=0\relax#2\else%
        \ifnum#1=1\relax#3\else%
            \ifnum#1=2\relax#4\else%
                \ifnum#1=3\relax#5\else#6\fi%
            \fi%
        \fi%
    \fi%
}
%    \end{macrocode}
% \end{macro}
% \end{macro}
% \end{macro}

% \begin{macro}{\@ordrel@util@decorator@run}
%    Sets up \cs{ARG} macro and calls the given handler without arguments inside math mode.
%    \begin{arguments}
%        \item No-arg handler expecing \cs{ARG} set
%        \item Argument to handler
%    \end{arguments}
%
%    \begin{macrocode}
\DeclareDocumentCommand{\@ordrel@util@decorator@run}{m m}{%
    {\ensuremath{\def\ARG{#2}#1}}%
}
%    \end{macrocode}
% \end{macro}

% \begin{macro}{\@ordrel@util@invoke@with@default}
%    Passes either the first argument, the specified option value
%    if the former is \cs{NoValue} or \cs{NoValue} if the option
%    is void to the provided command.
%    \begin{arguments}
%        \item Argument value (may be \cs{NoValue})
%        \item Fallback option name
%        \item Command or expression to invoke
%    \end{arguments}
%
%    \begin{macrocode}
\DeclareDocumentCommand{\@ordrel@util@invoke@with@default}{m m m}{%
    \IfValueTF{#1}{#3{#1}}{%
        \ifoptionvoid{#2}%
            {#3{#1}}% #1 is NoValue
            {#3{\option{#2}}}%
    }%
}
%    \end{macrocode}
% \end{macro}

% \subsection{User-exposed commands}

% \begin{macro}{\ordrel}
%    The main command provided to the end user. Takes optional
%    list of arguments and mandatory relation label followed by
%    optional subscript and/or superscript embellishments.
%    \begin{syntax}
%        \cs{ordrel}\meta{*}\oarg{options}\Arg{label}\meta{\_}\meta{sub}\meta{\^}\meta{sup}
%    \end{syntax}
%    \begin{arguments}
%        \item Optional star
%        \item List of options
%        \item The relation label
%        \item The label subscript embellishment
%        \item The label superscript embellishment
%    \end{arguments}
%    Star effectively adds |spacing=no| option to the option list.
%    \begin{macrocode}
\DeclareDocumentCommand{\ordrel}{s O{} m e_ e^}{%
    \begingroup%
        \@ordrel@util@options{#2}{#1}%
        \ensuremath{\@ordrel@impl{#3}{#4}{#5}}%
    \endgroup%
}
%    \end{macrocode}
% \end{macro}

% \begin{macro}{\NewOrdRel}
%    Defines new \cs{ordrel}-like command with the given
%    list of default options. The new command takes the same
%    arguments as \cs{ordrel} except for case when the starred
%    version of this command was used to define it with explicitly
%    provided label.
%    \begin{syntax}
%        \cs{NewOrdRel}\oarg{options}\Arg{command}
%        \cs{NewOrdRel}\meta{*}\oarg{options}\Arg{label}\Arg{command}
%    \end{syntax}
%    The commands validates option list before proceeding
%    with new command definition.
%    \begin{macrocode}
\DeclareDocumentCommand{\NewOrdRel}{s O{}}{%
    % check if options are correct early
    {\options{/ordrel,#2}}%
    \IfBooleanTF{#1}%
        {\@ordrel@new@label{#2}}%
        {\@ordrel@new@nolabel{#2}}%
}
%    \end{macrocode}
% \end{macro}

% \begin{macro}{\NewOrdRelX}
%    Defines new \cs{ordrel}-like command with the given
%    list of default options. The new command takes the same
%    arguments as \cs{ordrel} except for case when the starred
%    version of this command was used to define it with explicitly
%    provided label.
%    \begin{syntax}
%        \cs{NewOrdRel}\oarg{options}\Arg{command}\meta{\_}\meta{sub}\meta{\^}\meta{sup}
%        \cs{NewOrdRel}\meta{*}\oarg{options}\Arg{label}\Arg{command}\meta{\_}\meta{sub}\meta{\^}\meta{sup}
%    \end{syntax}
%    Unlike \cs{NewOrdRel}, this command allows to specify
%    default values for subscript and superscript for the
%    newly defined command in somewhat more convenient way.
%
%    The commands validates option list before proceeding
%    with new command definition.
%    \begin{macrocode}
\DeclareDocumentCommand{\NewOrdRelX}{s O{}}{%
    % check if options are correct early
    {\options{/ordrel,#2}}%
    \IfBooleanTF{#1}%
        {\@ordrel@new@label@x{#2}}%
        {\@ordrel@new@nolabel@x{#2}}%
}
%    \end{macrocode}
% \end{macro}

% \begin{macro}{\SetupOrdRel}
%    Allows to configure current scope-level default options.
%    \begin{syntax}
%        \cs{SetupOrdRel}\Arg{options}
%        \cs{SetupOrdRel}*\Arg{options}
%    \end{syntax}
%
%    The command first validates the given options, then
%    modifies the scope-level variable adding (or assigning
%    in case of starred version) the passed options to it.
%
%    \begin{macrocode}
\DeclareDocumentCommand{\SetupOrdRel}{s m}{%
    % validate options
    {\options{/ordrel,#2}}%
    \IfBooleanTF{#1}{%
        \def\@ordrel@options@global{%
            \optionsalso{#2,}%
        }%
    }{%
        \expandafter\def\expandafter\@ordrel@options@global\expandafter{%
            \@ordrel@options@global%
            \optionsalso{#2,}%
        }%
    }%
}
%    \end{macrocode}
% \end{macro}

%
% \iffalse
%</package>
% \fi
%
% \Finale
\endinput
